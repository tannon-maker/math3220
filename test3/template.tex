\documentclass{report}

\input{preamble}
\input{macros}
\input{letterfonts}

\title{\Huge{Math 3220}\\Midterm 3}
\author{\huge{Tannon Warnick}}
\date{\today}

\begin{document}
\maketitle

\qs{}{
Let \(f\) be a function on \(\bbR^2\) defined by \[f(x,y)= 
\begin{cases}
  0 &\text{ if } (x,y)=(0,0)\\
  \frac{xy}{x^2+y^2} &\text{ if } (x,y)\neq(0,0)\\
\end{cases}
\]
Prove that \begin{itemize}
  \item[\((i)\)] \(f\) is not continuous at \((0,0)\);
  \item[\((ii)\)] The first partial derivatives of f exist at every point of \(\bbR^2\). Is \(f\) differentiable at \((0,0)\)? Explain your answer!
\end{itemize}
}

\begin{itemize}
  \item[(\(i\))] To test continuity at \((0,0)\), we evaluate the limit along different paths:

  Along \(x = 0\):
  \[
  f(0,y) = 0 \Rightarrow \lim_{(x,y)\to(0,0)} f(x,y) = 0
  \]

  Along \(x = y\):
  \[
  f(x,x) = \frac{x^2}{2x^2} = \frac{1}{2} \Rightarrow \lim_{(x,y)\to(0,0)} f(x,y) = \frac{1}{2}
  \]

  Since the limits along different paths are not equal, the limit does not exist. Therefore, \(f\) is not continuous at \((0,0)\).

  \item[(\(ii\))] For \((x,y) \neq (0,0)\), we compute the partial derivatives using the quotient rule:

  \[
  \frac{\partial f}{\partial x} = \frac{y(x^2 + y^2) - xy(2x)}{(x^2 + y^2)^2} = \frac{y(y^2 - x^2)}{(x^2 + y^2)^2}
  \]

  \[
  \frac{\partial f}{\partial y} = \frac{x(x^2 + y^2) - xy(2y)}{(x^2 + y^2)^2} = \frac{x(x^2 - y^2)}{(x^2 + y^2)^2}
  \]

  At \((0,0)\), we compute the partial derivatives directly using the definition:

  \[
  \frac{\partial f}{\partial x}(0,0) = \lim_{h\to 0} \frac{f(h,0) - f(0,0)}{h} = \lim_{h\to 0} \frac{0}{h} = 0
  \]

  \[
  \frac{\partial f}{\partial y}(0,0) = \lim_{h\to 0} \frac{f(0,h) - f(0,0)}{h} = \lim_{h\to 0} \frac{0}{h} = 0
  \]

  So the partial derivatives exist everywhere, including at \((0,0)\). However, the partial derivatives are not continuous at \((0,0)\). For example, along the path \(x = y\):

  \[
  \frac{\partial f}{\partial x}(x,x) = \frac{x(x^2 - x^2)}{(2x^2)^2} = 0
  \]

  But along \(x = 0\):

  \[
  \frac{\partial f}{\partial x}(0,y) = \frac{y(y^2 - 0)}{y^4} = \frac{1}{y}
  \]

  which diverges as \(y \to 0\). Therefore, the partial derivatives are not continuous at \((0,0)\), and \(f\) is not differentiable at that point.
\end{itemize}
\qs{}{
If \(F\) is a differentiable real function defined in a convex open set \(U \subset \mathbb{R}^{n}\), such that \(\partial_{1} F(x)=0\) for every \(x \in U\), prove that \(F\) depends only on \(x_{2}, \ldots, x_{n}\).\\
}


Let \(\gamma(t) = (t, x_2, \ldots, x_n)\), which lies in \(U\) for all \(t\) in an interval since \(U\) is convex and open.

Define \(h(t) = F(\gamma(t)) = F(t, x_2, \ldots, x_n)\). Then:
\[
h'(t) = \partial_1 F(t, x_2, \ldots, x_n) = 0
\]
So \(h(t)\) is constant. Therefore, \(F(x_1, x_2, \ldots, x_n)\) does not depend on \(x_1\). Hence, there exists a function \(G\) such that:
\[
F(x_1, x_2, \ldots, x_n) = G(x_2, \ldots, x_n)
\]

\qs{}{

Let \(F: \mathbb{R}^{2} \longrightarrow \mathbb{R}^{2}\) be a map given by \(F=\left(F_{1}, F_{2}\right)\) where

\[
F_{1}(x, y)=e^{x} \cos (y) \quad \text { and } \quad F_{2}(x, y)=e^{x} \sin (y)
\]

for any \((x, y) \in \mathbb{R}^{2}\). Then:
\begin{itemize}
  \item[\((i)\)] Find the image of \(F\).\\
  \item[\((ii)\) ]Calculate the derivative \(F^{\prime}(x, y)\) and show that it is invertible at any point in \(\mathbb{R}^{2}\).\\
\end{itemize}
Thus, by the inverse function theorem, \(F\) is locally invertible, i.e., for any \((x, y) \in \mathbb{R}^{2}\) there are open neighborhoods \(U\) of \((x, y)\) and \(V\) of \(F(x, y)\) such that \(F: U \longrightarrow V\) is a bijection.

Show that \(F\) is not a bijection globally, i.e, \(F\) is not a bijection of \(\mathbb{R}^{2}\) onto the image of \(F\).
}

\begin{itemize}
  \item[(\(i\))] 
 To find the image of \(F(x, y) = (e^x \cos y, e^x \sin y)\), we observe that this has a natural polar form.

Let \(r = e^x > 0\), then we can rewrite:
\[
F(x, y) = (r \cos y, r \sin y)
\]

This is the standard polar coordinate representation of a point in \(\mathbb{R}^2\) with radius \(r\) and angle \(y\). Since:
\begin{itemize}
  \item \(e^x\) takes all positive real values as \(x \in \mathbb{R}\),
  \item and \((\cos y, \sin y)\) traces out the unit circle as \(y\) varies over \(\mathbb{R}\),
\end{itemize}
we see that \(F(x, y)\) traces out all points in \(\mathbb{R}^2\) except the origin.

Therefore, the image of \(F\) is:
\[
  F(\bbR^2) = \mathbb{R}^2 \setminus \{(0,0)\}
\]  \item[(\(ii\))] The Jacobian matrix is:
  \[
  F'(x,y) = 
  \begin{pmatrix}
    \frac{\partial F_1}{\partial x} & \frac{\partial F_1}{\partial y} \\
    \frac{\partial F_2}{\partial x} & \frac{\partial F_2}{\partial y}
  \end{pmatrix}
  =
  \begin{pmatrix}
    e^x \cos y & -e^x \sin y \\
    e^x \sin y & e^x \cos y
  \end{pmatrix}
  \]

  Its determinant is:
  \[
  \det(F') = e^{2x} (\cos^2 y + \sin^2 y) = e^{2x} > 0
  \]
  So \(F'\) is invertible everywhere, and by the inverse function theorem, \(F\) is locally invertible.

  However, \(F\) is not globally injective because:
  \[
  F(x, y + 2\pi) = F(x, y)
  \]
  So \(F\) is not globally a bijection onto its image.
\end{itemize}


\qs{}{

Let \(f\) be a function on \(\mathbb{R}\) defined by

\[
f(x)=x+2 x^{2} \sin \left(\frac{1}{x}\right)
\]

for \(x \neq 0\) and \(f(0)=0\). Show that
\begin{itemize}
  \item[\((i)\)] \(f\) is continuous on \(\mathbb{R}\);
  \item[\((ii)\)] \(f\) is differentiable on \(\mathbb{R}\);
  \item[\((iii)\)] the derivative \(f^{\prime}\) is not continuous at 0 ;
  \item[\((iv)\)] \(f^{\prime}(0)=1\);
  \item[\((v)\)] for any \(\epsilon>0\), the restriction of \(f\) to \((-\epsilon, \epsilon)\) is not injective.

\end{itemize}
This shows that, even for \(n=1\), the conclusions of inverse function theorem do not hold if \(f^{\prime}\) is not continuous.
\nt{
Hint: To prove (v), first show that a continuous function \(f\) cannot be injective in neighborhoods of local maxima and minima.

These must be critical points of \(f\), i.e. zeros of \(f^{\prime}\).\\
Then show that for every \(\epsilon>0\) the interval \((-\epsilon, \epsilon)\) contains infinitely many critical points of \(f\).

A critical point \(x\) of \(f\) is a maximum or minimum if \(f^{\prime \prime}(x) \neq 0\).\\
Therefore, it is enough to show that there is an \(\epsilon>0\) such that there are no \(x \in(-\epsilon, \epsilon)\) such that \(f^{\prime}(x)=0\) and \(f^{\prime \prime}(x)=0\).

To prove this observe that the derivatives \(f^{\prime}\) and \(f^{\prime \prime}\) are linear functions in \(A=\sin \left(\frac{1}{x}\right)\) and \(B=\cos \left(\frac{1}{x}\right)\) with coefficients which are rational functions in \(x\). Therefore, the equations \(f^{\prime}(x)=0\) and \(f^{\prime \prime}(x)=0\) are a linear system of two equations for \(A\) and \(B\) with rational function coefficients.

Explicitly solve this system for \(A\) and \(B\). Then calculate \(A^{2}+B^{2}\). From the result you should see that for small \(x\) this expression cannot be 1 , contradicting the choice of \(A\) and \(B\). Therefore, for small \(x, f^{\prime}\) and \(f^{\prime \prime}\) cannot simultaneously vanish at \(x\).
}}
Let
\[
f(x) =\begin{cases} x + 2x^2 \sin\left(\frac{1}{x}\right)&\text{ if }x \neq 0\\ 0&\text{ if } x=0\end{cases}
\]

\begin{itemize}
  \item[(\(i\))] As \(x \to 0\),
  \[
  \left|2x^2 \sin\left(\frac{1}{x}\right)\right| \leq 2x^2 \to 0
  \]
  So \(f(x) \to 0\), implying continuity at 0. \(f\) is continuous on \(\mathbb{R} \setminus \{0\}\cup \{0\} = \bbR\)

  \item[(\(ii\))] For \(x \neq 0\),
  \[
  f'(x) = 1 + 4x \sin\left(\frac{1}{x}\right) - 2 \cos\left(\frac{1}{x}\right)
  \]
  At \(x = 0\):
  \[
  f'(0) = \lim_{h \to 0} \frac{f(h) - f(0)}{h} = \lim_{h \to 0} \left(1 + 2h \sin\left(\frac{1}{h}\right)\right) = 1
  \]
  So \(f\) is differentiable everywhere.

  \item[(\(iii\))] \(f'(x)\) oscillates as \(x \to 0\), due to the presence of \(\cos(1/x)\). Hence, \(f'\) is not continuous at 0.

  \item[(\(iv\))] As above, \(f'(0) = 1\).

  \item[\((v)\)] 
We want to show that for any \(\varepsilon > 0\), the function \(f\) is not injective on \((-\varepsilon, \varepsilon)\).

A function is not injective in any neighborhood that contains a local maximum or minimum. Local extrema occur at critical points, i.e., points where \(f'(x) = 0\), and they are true extrema if also \(f''(x) \neq 0\). Therefore, it suffices to show that every open interval around 0 contains critical points that are local extrema.

For \(x \ne 0\), recall:
\[
f(x) = x + 2x^2 \sin\left(\frac{1}{x}\right)
\]
We compute its first and second derivatives:

\[
f'(x) = 1 + 4x \sin\left(\frac{1}{x}\right) - 2 \cos\left(\frac{1}{x}\right)
\]

\[
f''(x) = 4 \sin\left(\frac{1}{x}\right) - \frac{4}{x} \cos\left(\frac{1}{x}\right) + \frac{2}{x^2} \sin\left(\frac{1}{x}\right)
\]


The function \(f'(x)\) is oscillatory near 0 due to the sine and cosine terms, and changes sign infinitely often as \(x \to 0\). Therefore, there exist infinitely many values of \(x\) arbitrarily close to 0 where \(f'(x) = 0\). These are critical points.

Suppose \(f'(x) = 0\) and \(f''(x) = 0\). Then we would have a system of two equations in \(A = \sin\left(\frac{1}{x}\right)\), \(B = \cos\left(\frac{1}{x}\right)\) with rational function coefficients:

\[
\begin{cases}
1 + 4x A - 2 B = 0 \\
4 A - \frac{4}{x} B + \frac{2}{x^2} A = 0
\end{cases}
\]

We can solve this system for \(A\) and \(B\), and compute \(A^2 + B^2\). However, for small \(x\), the value of \(A^2 + B^2\) is not equal to 1, which contradicts the identity \(\sin^2(\cdot) + \cos^2(\cdot) = 1\). Therefore, for small \(x\), \(f'(x)\) and \(f''(x)\) cannot simultaneously vanish.

There are infinitely many critical points near 0, and at least some of them must be local maxima or minima. Hence, for any \(\varepsilon > 0\), the function \(f\) has local extrema in \((-\varepsilon, \varepsilon)\), so \(f\) cannot be injective on that interval.
\end{itemize}
\end{document}
